\section*{前言}
\zihao{-4}
\label{section:before}

呼叫中心(Call Center)又称为客户服务中心,是基于CTI技术\wordnote{
    CTI技术传统的定义为“计算机电话集成”,即Computer Telephony Integration。
    随着电信通信技术的发展,现在的定义为“计算机电信集成”,即Computer Telecommunication Integration。
},利用互联网、电信通信网络和计算机网络的多像功能集成,并与企业连为一体的完整的综合信息服务系统。
早期的呼叫中心依赖电话机或排队机,实现客户电话的纯人工接听。
随着计算机技术、通信技术和电子商务模式的发展,呼叫中心呈现多媒体化、分布式的发展趋势。
现今的呼叫中心架构于IP协议与计算机网络之上,将之前的电话呼叫转变为包括邮件、电话和基于Web2.0技术的在线文本交谈等多种形式,
成为包含通信、计算、管理和业务支撑特性的全媒体交互中心(\citet{马晓军-1})。

然而,新的技术和商业模式在给企业创造利润、引领企业进行销售与服务模式变革的同时,也给企业带来了新的挑战。
以阿里巴巴淘系为例,包含淘宝、咸鱼等在内的电商平台具有亿量级的在线商品库,用户数同样在数亿规模。
每天淘系平台会产生包含购物、物流和售后咨询在内的海量服务请求。
即便具有现代的呼叫中心,简单但繁复的问题咨询仍给企业带来了庞大的人工成本。

因此本文从呼叫中心的核心功能即客服问答出发,结合机器学习等人工智能技术、方法、模型和算法,
分析和设计一种智能客服问答机器人(以下简称智能客服),以帮助企业降低人工客服工作量和提高服务资源使用效率。

在此本文题目要求为:“呼叫中心的AI接线员的系统的设计与实现”,并有以下三点具体要求:

\begin{enumerate}[leftmargin=3.5em,itemindent=0em,label=(\arabic*),itemsep=0pt,topsep=0pt]
    \item[\textasteriskcentered] 呼叫中心的AI接线员的系统的设计与实现
    \item 
    调研目前国内外最好的三家(款)系统。对比它们的功能、性能和可用性等相关指标,指出它们的优缺点;
    \item 
    完成总结报告和汇报PPT,解释其主要工作原理和实现方式(如何训练模型和算法、如何实现应用),附演示视频;
    \item 
    提出具体研究和实施方案,
    即如何用机器学习等人工智能的技术、方法、模型和算法来实现面向特定领域的对话机器人
    (如替代电信114接线员)。
\end{enumerate}

基于时间因素和论文篇幅,本文结合实际情况对要求进行了适当松弛,并对相关概念进行了界定。

基于Web2.0技术的在线文本交谈已成为呼叫中心客服与用户交互的主要形式。
虽然电话和邮件等形式通过语音识别、分句等处理也可转化为文本交谈,
但其属于另外相对独立的研究领域,不在本文研究范畴。
因此本文并不考虑呼叫中心的电话、邮件等交互形式,
而以在线文本交谈为情景,设计和实现智能客服问答机器人即“AI接线员”。

呼叫中心的智能客服目前尚处发展阶段,缺乏统一和公认的评价体系。
因此本文无法界定“最好”,在此通过经验确定了相关评价指标,挑选和对比了国内外三家具有代表性的此类系统。

由于市面上具有代表性的智能客服系统尚未开源,因此无法准确得知其原理细节和实现方式。
在此,本文通过官网展示或演示视频,结合现有研究,对其工作原理和实现方式进行合理推测。

综上所述,本文以呼叫中心在线文本交谈为情景,
调研了国内外三家具有代表性的客服智能问答系统,建立了相关评价指标进行对比,
推断了其主要工作原理和实现方式,并具体提出了一种面向客服领域的智能问答机器人。
