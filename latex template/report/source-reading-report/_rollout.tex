\section{对话推演}
\label{section:rollout}
\zihao{-4}

小节\ref{subsection:decoding}中所提的Likelihood-based decoding具有一定的缺陷。
在谈判过程时,代理有两类策略,一是接受对方的要求,二是给出自己的要求,即\quotes{Counter Offer}。
接受要求这一行为在Likelihood-based decoding中会具有更高的概率,
因为接受要求比给出要求更易达成\quotes{Deal}。

本节中所提出的Goal-based decoding将允许更多的谈判策略。
例如,在考虑项目价值和分数之后,代理可能会采用欺骗的手段\wordnote{
    在人-人谈判训练集中,人的表述往往是诚实的,能够较直观的反应自己的需求。
}的获取较高价值的项目,从而拿到更高的分数。

\begin{figure*}[ht!]
\centering
% Declare layers
%\pgfdeclarelayer{background}
%\pgfsetlayers{background,main}

\begin{tikzpicture}[scale=.8,cap=round]
\tikzstyle{them} = [rectangle, rounded corners, minimum width=2cm, text width=2.6cm, minimum height=1cm, draw=black, fill=red!30]
\tikzstyle{candidate} = [rectangle, rounded corners, minimum width=2cm, text width=3cm, minimum height=0.3cm, draw=black, fill=blue!30]
\tikzstyle{rollout} = [rectangle, rounded corners, minimum width=3.4cm, minimum height=0.3cm, text width=3.4cm, draw=black, fill=red!30]
\tikzstyle{score} = [diamond, minimum width=0.5cm, minimum height=0.5cm, draw=black, fill=green!30, inner sep=0,outer sep=0]



\tikzstyle{arrow} = [thick,->,>=stealth]


\node (start) [them] {\them You get one book and I'll take everything else.};
\node (candidate2) [candidate, right of=start, xshift=2.6cm, yshift=0.75cm] {
\you Great deal, thanks!};
\node (candidate1) [candidate, right of=start, xshift=2.6cm, yshift=-0.75cm] {\you No way, I need all 3 hats};
%\node (candidate3) [candidate, below of=candidate2, yshift=-1cm] {Start2};

\node (rollout11) [rollout, right of=candidate1, xshift=3cm, yshift=-0.4cm] {\them Ok, fine};
\node (rollout12) [rollout, right of=candidate1, xshift=3cm, yshift=0.4cm] {\them I'll give you 2};

\node (rollout21) [rollout, right of=candidate2, xshift=3cm, yshift=-0.4cm] {\them No problem};
\node (rollout22) [rollout, right of=candidate2, xshift=3cm, yshift=0.4cm] {\them Any time};


\node (choose11) [candidate, right of=rollout11, xshift=3cm, text width=2.7cm] {\textbf{choose:} 3x \textit{hat}};
\node (choose12) [candidate, right of=rollout12, xshift=3cm, text width=2.7cm] {\textbf{choose:} 2x \textit{hat}};

\node (choose21) [candidate, right of=rollout21, xshift=3cm, text width=2.7cm] {\textbf{choose:} 1x \textit{book}};
\node (choose22) [candidate, right of=rollout22, xshift=3cm, text width=2.7cm] {\textbf{choose:} 1x \textit{book}};


\node (score11) [score, right of=choose11, xshift=1.5cm] {9};
\node (score12) [score, right of=choose12, xshift=1.5cm] {6};

\node (score21) [score, right of=choose21, xshift=1.5cm] {1};
\node (score22) [score, right of=choose22, xshift=1.5cm] {1};



\draw [arrow] (start) -- (candidate1);
\draw [arrow] (start) -- (candidate2);


\foreach \i in {1,2}
{
\draw [arrow] (candidate1) -- (rollout1\i);
\draw [arrow] (candidate2) -- (rollout2\i);

\foreach \j in {1,2}
{
\draw [arrow] (rollout\i\j) -- (choose\i\j);
\draw [arrow] (choose\i\j) -- (score\i\j);
}
}

\draw [
    thick,
    decoration={
        brace,
        mirror,
        raise=1.85cm,amplitude=6pt
    },
    decorate
] (start.west) -- (start.east) node [pos=0.5,anchor=north,yshift=-1.9cm] {Dialogue history}; 

\draw [
    thick,
    decoration={
        brace,
        mirror,
        raise=1.10cm,amplitude=6pt
    },
    decorate
] (candidate1.west) -- (candidate1.east) node [pos=0.5,anchor=north,yshift=-1.15cm] {Candidate responses}; 

\draw [
    thick,
    decoration={
        brace,
        mirror,
        raise=1.5cm,amplitude=6pt
    },
    decorate
] (rollout12.west) -- (choose12.east) node [pos=0.5,anchor=north,yshift=-1.55cm] {Simulation of rest of dialogue}; 

\draw [
    thick,
    decoration={
        brace,
        mirror,
        raise=1.5cm,amplitude=4pt
    },
    decorate
] (score12.west) -- (score12.east) node [pos=0.5,anchor=north,yshift=-1.55cm] {Score}; 


\end{tikzpicture}
\caption{
\label{figure:rollouts} 
对话推演
}\vspace{-2.5mm}
\end{figure*}
与强化学习不同,FAIR在本节利用代理行为概率$p_\theta$计算期望谈判分数,并且这里的计算考虑代理的未来行为,
如图\ref{figure:rollouts}。
FAIR通过两个阶段来实现对话推演。首先,以$U=u_{0..c}$表示整个谈判过程中的表述\wordnote{
    utterances,包括已出现和可能出现的表述。
},以$x_{0..n-1}$表示当前的对话历史。
通过计算$x_{n+k+1,T}$对应的$p_\theta$来获取$u=x_{n,n+k}$。
再通过计算$u=x_{n,n+k}$对应的$R(u)$来挑选最优的输出选择$\vec{o}$。
因此该方法下的期望价值分数与对话中的表述相关:
\begin{equation}
R(x_{n.. n+k})=\mathbb{E}_{x_{(n+k+1.. T;\vec{o})}\sim p_\theta}[r(\vec{o})p_\theta(\vec{o}|x_{0..T})]
\end{equation}

接下来返回最大期望价值分数对应的表述:
\begin{equation}
u^{*}=\argmax_{u\in U}R(u)
\end{equation}

FAIR的对话推演为5次,预测轮数为10轮。