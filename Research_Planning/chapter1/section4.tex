\section{实验及结果分析}\label{sec:1.4}
\subsection{数据收集和预处理}\label{subsection:1.3.4}

为了测试抱怨问题识别方法的准确性,本文从新浪微博随机选取了281条与中国移动相关的抱怨微博作为实验数据。
由于这些数据存在相当多的噪声,在使用前需要进行一定的预处理,以消除不必要的干扰。预处理的过程如下:

(1)去除散列标签。抱怨微博文本中存在很多散列标签(如URL、“\#”、“@”),
它们又是作为句子的一部分存在,而又是只作为引用标签,
本步骤从抱怨微博文本中通过正则匹配去除无意义的散列标签,留存那些作为句子成分和陈述了抱怨问题的标签;

(2)还原简写。由于微博具有自由编辑的特性,微博用户在发表微博抱怨时可能使用简写,
如“移动”和“中移动”都是“中国移动”的简写。尽管人们可以很轻易地分析和理解这些简写,
但是实现自动地分析和理解它们不在本文讨论范畴。为处理这些简写,
本步骤通过人工构建简写还原词库,在避免歧义句的基础上(如“从中移动开”),将它们还原为“中国移动”;

(3)去除重复符号和表情符号。为了实现多样化的表达,
用户可以使用重复符号和表情符号(如“::>\_<::”)来表达特定的情绪。
由于本文主要通过语义和语法结构分析进行抱怨识别研究,
不涉及抱怨微博所需的情感信息,不需要对重复符号和表情进行分析,故在此去除重复符号和表情符号;

(4)分句分词。考虑到抱怨微博文本通常由多个句子组成,
为了保证句子组成分析和依存分析的准确性,实验对抱怨微博文本进行了分句和分词处理。

在完成了必要的预处理后,实现从281条抱怨微博中获取到了756个句子,
其中318个句子包括至少一个抱怨问题,而438个句子没有包含抱怨问题,具体的实验数据统计情况如表3所示。
\begin{table}[ht]
    \centering
    \caption{实验数据统计}\label{tab:1.3}
    \vskip -10pt
    \begin{tabularx}{\textwidth}{XY}
    \toprule
    实验数据 & 数量 \\
    \midrule
    微博数量 & 281 \\
    句子数目 & 756 \\
    微博句子平均数 & 2.69 \\
    不存在抱怨问题的句子数目 & 438 \\
    存在抱怨问题的句子数目 & 318 \\
    问题句子涉及抱怨目标短语数目 & 735 \\
    问题句子设计候选触发核心词数目 & 1363 \\
    \bottomrule
    \end{tabularx}
\end{table}

实验以318个至少包含一个抱怨问题的句子为输入,
通过完成触发目标短语识别、触发核心词识别和抱怨问题路径的抽取三个任务实现抱怨问题的识别。
在触发核心词识别实验中,实验通过句法分析从318个句子中获取1363个与抱怨目标短语存在依存关系的词语,
其中249个句子中的1040个词语作为训练集,69个句子中的323个词语作为测试集。为评价\autoref{alg:1.1},
实验基于现有的知识库信息和专家意见,采用内容编码的方式对实验数据进行了触发核心词和抱怨问题的标注。

\subsection{实验设计和评价指标}\label{subsection:1.4.2}

实验将触发核心词的识别问题转化为分类问题,并选用LibSVM包完成分类任务。
在进行抱怨问题路径抽取和触发核心词识别实验之前,
先构建描述抱怨产品及特征的词库,并基于该词库进行抱怨目标短语识别。

实验中,负面和否定词词库一部分是根据HowNet提供的3116个中文负面情感词语和1254个负面评价词语构建,
另一部分是通过收集如“不”、“没”等常用否定词构建;
此外,根据移动通信行业特征对词库进行了扩充,增加了如“慢”、“断”、“篡改”和“偷偷”之类的词语。

在抱怨问题的抽取中,句法分析以识别的抱怨目标短语及其对应触发核心词为核心,
获取满足特定条件的抱怨目标短语和触发核心词所在的组成分析树,以实现抱怨问题的抽取。

在触发核心词识别实验中,先通过Standard Parser进行句法分析,获取与抱怨目标短语存在依存关系的词语,
将这些词作为候选触发核心词,再根据句法分析、否定和负面词库获取核心词的位置特征
($Tri_L$)、语法特征($Tri_Syn$)和统计特征($Tri_Sta$),
实验通过这三类特征的7种不同组合
$Tri_L$、$Tri_Syn$、$Tri_Sta$、$Tri_{L\&Syn}$、$Tri_{L\&Sta}$、$Tri_{Syn\&Sta}$和$Tri_{L\&Syn\&Sta}$
实现触发核心词的分类,并通过结果分析选取效果最好的组合。

关于评价指标,实验采用标准机器学习分类评价指标(准确率、精确率、召回率和$F_1$值)\cite{cortes1995support}对抱怨问题的抽取和触发核心词的识别进行评价。
准确率$A$是对总体识别效果进行评价,通过识别结果中正确识别的个体与所有个体的比值来计算;
精确率$P$表示识别方法(识别触发核心词和抱怨问题)的精确性,它通过正确识别为某类别的个体数量$TP$与被识别为该类别的个体总数的比值来计算;
召回率$R$表示识别方法的完整性,它通过正确识别为某类别的个体数量$TP$与该类别下个体总数的比值来计算;
$F_1$值是精确率和召回率的调和平均值。各评价指标的具体公式如下:
\begin{align}
    A &=(TP+TN)/(TP+TN+FP+FN)\\ 
    P &=TP/(TP+FP)\\
    R &=TP/(TP+FN)\\
    F_1 &=(2PR)/(P+R)
\end{align}

本实验中,$TP$表示正确识别出的触发核心词总数;
$TN$表示正确识别出的非触发核心词总数;
$FP$表示不能构成触发核心词却被识别为触发核心词的总数;
$FN$表示可以构成触发核心词却被识别为非触发核心词的总数。


\subsection{实验结果和分析}\label{subsection:1.4.3}
根据上述实验设计和评价指标设定,触发核心词识别实验的结果如\autoref{tab:1.4}所示。
\begin{table}[ht]
    \centering
    \caption{触发核心词识别实验结果及对比}\label{tab:1.4}
    \vskip -10pt
    \begin{tabularx}{\textwidth}{lYYYYYYY}
    \toprule
    算法 & $Tri_L$ & $Tri_{Syn}$ & $Tri_{Sta}$ & $Tri_{L\&Syn}$ & $Tri_{L\&Sta}$ & $Tri_{Syn\&Sta}$ & $Tri_{L\&Syn\&Sta}$ \\
    \midrule
    准确率 & 0.7492 & 0.8824 & 0.7802 & 0.8142 & 0.7957 & 0.7956 & 0.8111 \\
    精确率 & 0 & 0.8644 & 0.6190 & 0.7839 & 0.7143 & 0.6667 & 0.7500 \\
    召回率 & 0 & 0.6296 & 0.3210 & 0.3580 & 0.3086 & 0.3704 & 0.3704 \\
    $F_1$ & 0 & 0.7286 & 0.4228 & 0.4915 & 0.4130 & 0.4762 & 0.4959 \\
    \bottomrule
    \end{tabularx}
\end{table}

由\autoref{tab:1.4}可知,在触发核心词的识别实验中只考虑语法特征的触发核心词$Tri_{Syn}$分类模型效果最佳,
这说明了模型$Tri_{Syn}$在触发核心词识别中具有明显优势。
在单类特征模型中,$Tri_{Syn}$效果最佳,$Tri_{Sta}$次之,$Tri_L$最差;
在多类特征模型中,考虑位置特征、语法特征和统计特征的$Tri_{L\&Syn\&Sta}$的效果最佳,
但与$Tri_{L\&Syn}$、$Tri_{L\&Sta}$、$Tri_{Syn\&Sta}$的效果相差不大,
且四个多类模型的触发核心词识别效果皆优于单类特征模型$Tri_L$和$Tri_{Syn}$但皆劣于$Tri_{Syn}$,
这说明语法特征能够显著提高触发核心词的识别效果,尽管位置特征和统计特征也可以用于触发核心词的识别,
但其效果明显低于语法特征。基于上述触发核心词识别实验结果对比,
实验采用效率最高的模型$Tri_{Syn}$进行触发核心词的识别,
并基于抱怨目标短语和识别的触发核心词进行抱怨问题路径抽取,进而完成对抱怨问题的识别。

综上所述,本文进行的实验有如下发现:语法特征对提高抱怨问题触发核心词识别的准确性具有显著作用;
抱怨目标短语和抱怨问题触发核心词的简单组合不能很好地表示抱怨问题,考虑两者的句法关系可以提高抱怨问题识别的准确性。