\chapter{基于支持向量机的在线负面口碑处理专家识别方法}\label{chapter:2}
\section{引言}\label{sec:2.1}

社会化媒体平台已成为企业客户发表和传播在线负面口碑的主要渠道,
数量庞大、分布广泛的在线负面口碑给企业运营和管理带来了极大的挑战。
若不能有效处理在线负面口碑使其负面影响降到最低,不仅会给企业业务带来压力,
而且会对企业声誉产生不可逆转的消极影响,造成巨大损失。
如2017年4月,美联航因机票超售通知航警将一位69岁的华裔医生暴力带离航班,
事件被同机旅客拍下并上传至网络,
引起大范围讨论,次周周二美联航股价一度跌超4\%,成为标准普尔500指数中表现最差的股票\cite{victor2017airlines}。

消费者抱怨行为是用户在购买企业产品或服务过程中感知不满意引发的一系列行动。
其采取的行动包括通过企业官网、客户电话等直接向企业投诉、向第三方机构申诉和通过社会化媒体平台(如新浪微博、在线论坛等)发布或转发抱怨等。
在线负面口碑作为大量发布或被传播的抱怨行为已成为企业客户关系管理中的重要问题\cite{fornell1987defensive}。
社会化媒体平台的匿名性使得发布或传播负面口碑的用户一定程度地避免遭受社会后果或报复\cite{woong2011selective}。
这一特性使用户更愿意通过社会化媒体平台分享他们的负面体验。
相比传统的负面口碑,在线负面口碑会对消费者行为产生更加强烈的影响,
因此有效处理在线负面口碑对于企业具有重要的现实意义。

目前,企业大多采取人工客服的方式进行负面口碑的处理。
这种方式不仅成本高效率低,而且很多时候并不能很好的解决用户的抱怨问题进而无法有效处理负面口碑。
甚至企业这种主动干预、过度参与负面口碑处理的行为可能被用户怀疑而丧失自身公信力\cite{dellarocas2006strategic},
加深用户的厌恶感,导致更为强烈的在线负面口碑产生与扩散,造成企业的二次损失\cite{van2012online}。
同时由于抱怨问题的海量产生与在线负面口碑的迅速扩散,
仅依靠传统的人工客服已不能及时且有效地应对在线负面口碑。
有效处理在线负面口碑已成为企业管理领域的热门研究课题。

Noble等\cite{noble2012let}提出企业应建立虚拟社区并主动对接用户,监管和处理在线负面口碑。
然而社区普通用户通常不愿直接和企业沟通,
更愿在虚拟社区与其他用户交流或选择在社会化媒体平台上提出抱怨问题,进而发布和传播在线负面口碑。
诸多企业以开设在线客服和官方论坛的形式,
将客服平台迁移至社区和平台中,并寻求辅助处理系统帮助解决这一问题。
基于此背景,袁乾\cite{cai2016}采用影响力衰减函数描述负面口碑的多变趋势,
提出了一种基于回归树与衰减函数的IMM-RTDF模型来预测在线负面口碑的影响力,
为企业配置在线负面口碑处理资源提供决策支持。
即便如此,面对社会化媒体平台上海量发布和迅速传播的在线负面口碑,一线处理员工仍应接不暇。
与此同时,社会化媒体发展过程中的一种现象值得关注。
社会化媒体为用户提供了知识共享的平台,用户可在其中搜索、发布和回应信息,
与其他用户形成如关注、被关注等各种虚拟社会关系,并基于社会交换的目的与他人分享和交流知识,
获得一定的满足感。
随着用户数量、累计发布内容和互动的增多,少量具有更多知识的用户逐渐凸显,成为社区平台上的专家用户。
Cheng\cite{cheng2015exploiting}研究表明社区平台上绝大部分有价值的解答来源于这些少量的专家用户。
因此,若能识别出这些专家用户,并给予适当引导,企业可在扩展专家资源的同时,借助专家用户专业知识程度高、互动用户范围广,
更易被普通用户接受的优势显著提高服务补救质量,有效处理在线负面口碑。